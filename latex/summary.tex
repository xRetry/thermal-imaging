\documentclass[10pt,a4paper,german]{article}
\usepackage[utf8]{inputenc}
\usepackage{amsmath}
\usepackage{amsfonts}
\usepackage{amssymb}

\setlength\parindent{0pt}

\begin{document}
\section{Bildverarbeitung}
\subsection{JPEG Verarbeitung}
Die Aufnahmen der Kamera werden von der Software als JPEG Dateien abgespeichert.
Dies ist für eine weitere quantitative Auswertung problematisch, da die gemessenen Temperaturen in RGB-Werte umgewandelt werden.
Eine wichtige Aufgabe ist es daher, die ursprünglichen Temperaturinformationen für jeden Pixel möglichst genau wiederherzustellen.
Dies wird in zwei Schritten bewerkstelligt:
Einerseits der Zuordnung von Farb- und Temperaturwerten mittels des im Bild inkludierten Farbbalkens, andererseits der Umwandlung aller Pixel in Temperaturen.

\subsubsection{Farb-Temperatur Zuordnung}
Jedes Bild verfügt über einen Farbbalken, welcher zum visuellen Abschätzen der Temperaturen innerhalb des Bildes dient.
Der Maximal- und Minimalwert des Balkens ist jeweils mit der zugehörigen Temperatur gekennzeichnet.
Mit Hilfe dieser Informationen lässt sich eine Zuordnung von Farbwert und Temperatur erzeugen.
Für jeden Pixel $i$ des Farbbalkens kann die korrespondierende Temperatur $T_{scale,i}$ mittels folgender Formel berechnet werden:

\begin{equation}
    T_{scale,i} = \frac{i}{n} \left(T_{max} - T_{min}\right) + T_{min}
\end{equation}

wobei $n$ die Anzahl der Pixel des Farbbalkens representiert.
Die beiden angegebenen Randtemperaturen $T_{min}$ und $T_{max}$ werden jeweils ganzzahlig angegebenen.
Dies lässt die Frage offen, ob diese zur Präsentation auf- oder abgerundet wurden.
Die Standardunsicherheit der Randtemperaturen $u_{round}$ kann quantifiziert werden durch:

\begin{equation}
    u_{round} = \frac{1}{2\sqrt{3}}
\end{equation}

Folglich ergibt sich ein möglicher Fehler für die berechneten Temperaturen, welcher sich über die Fehlerfortpflanzung berechnen lässt:

\begin{equation}
    u_{scale,i} = \sqrt{\left(\frac{i}{n} u_{round} \right)^2 + \left(\left(1 - \frac{i}{n}\right) u_{round}\right)^2}
\end{equation}

Hier beschreibt $u_{scale,i}$ die Standardunsicherheit des berechneten Temperaturwerts der $i$-ten Balkenfarbe.

\subsubsection{Farbinterpolation}
Nun sind zwar die Temperaturen für alle Farben des Farbbalkens bekannt, jedoch können die Pixel des eigentlichen Bildes Farbwerte besitzen, die zwischen denen des Farbbalkens liegen.
Dies Bedarf dem Einsatz von Farbinterpolation, was jedoch keine triviale Aufgabe ist.
Der Farbverlauf des Farbbalkens stellt eine dreidimensionale Kurve im RGB-Raum dar, welche die Anwendung vieler der üblicher Interpolationsmethoden ausschließt.
Folglich viel unsere Wahl auf Nearest-Neighbor Interpolation, da alle anderen getesteten Methoden fehlschlugen.

\begin{equation}
    u_{ip} = \frac{T_{max} - T_{min}}{4\sqrt{3}n}
\end{equation}
\begin{equation}
    u_{pixel} = \sqrt{u_{scale,i}^2 + u_{ip}^2}
\end{equation}

\subsection{TIFF Verarbeitung}
Die Wahl der entsprechenden Einstellung in der Seek App ermöglicht es aufgenommene Bilder im TIFF Format abzuspeichern. 
Dieses Bildformat ist in der Lage mehrere Bilder in eine Datei zusammenzufassen.
Im Falle der Seek Thermal App befinden sich darunter auch die gemessenen unveränderten Temperaturwerte.
Dementsprechend entfällt die Notwendigkeit die Temperaturen von den Farbwerten der Pixel abzuleiten.
Der relative Fehler jedes Temperaturwertes ist in diesem Fall rein von der thermischen Messgenauigkeit der Kamera bestimmt.
Eine Internetrecherche für das Modell "Seek Compact XR" ergibt eine Temperaturtoleranz $tol_{th} = 0.07 ^{\circ}C$.
Die entsprechende Standardunsicherheit berechnet sich mit:

\begin{equation}
    u_{th} = \frac{tol_{th}}{2\sqrt{3}}
\end{equation}

\subsection{Segmentauswahl}
Zur Durchführung weiterer Auswertungen ist es notwendig Linien und Rechtecksausschnitte definieren zu können.
Da Rechtecke nicht in beliebiger Orientierung benötigt werden können diese einfach über zwei diagonale Eckpixel definiert werden.
Alle Temperaturwerte innerhalb dieser Auswahl werden zur weiteren Auswertung herangezogen.
\\
\\
Linien stellen dagegen eine größere Herausforderung dar, da eine beliebige Orientierung ermöglicht werden soll.
Für Linienpunkte die sich nicht genau mit den vorhandenen Pixeln überdecken wird daher lineare Interpolation verwendet, um passende Temperaturwerte zu erhalten.
Die Interpolation wird ebenfalls für die Standardunsicherheiten durchgeführt, welche für jeden Pixel bekannt sind.
Somit erhält jeder Punkt der Linie einen entsprechenden Temperatur- und Unsicherheitswert.

\section{Temperaturkalibrierung}
Mittels einer Referenzmessung können die absoluten Temperaturwerte der wahren Temperatur angepasst werden.
Hierzu wird ein Bild eines Referenzfläche mit bekannten Emissionsgrad aufgenommen.
Aus dem Bild werden alle Temperaturen vom Bereich der Referenzfläche ausgewählt und gemittelt.
Anschließend können die Temperaturwerte des gesamten Bildes um die Differenz zwischen Mittelwert $\overline T_{sel}$ und Referenztemperatur $T_\textit{ref}$ verschoben werden.

\begin{equation}
    T_{pixel,cal} = T_{pixel} - \left( \overline T_\textit{sel} - T_\textit{ref} \right)
\end{equation}

In die Betrachtung der Fehlerfortpflanzung fließen hier die Standardunsicherheiten aller Temperaturen ein.
Die Werte des Bildes innerhalb des Vergleichbereichs zeigen dabei eine gewisse Temperaturverteilung, welche im Normalfall einer Gauß-Verteilung entspricht.
Zusätzlich besitzen die einzelnen Temperaturen die diese Verteilung bilden selbst über ihren eigenen Unsicherheitswert.
Folglich ergibt sich für die Standardunsicherheit $u_{sel}$ der gemittelten Bereichstemperatur $T_{sel}$ eine kombinierte Unsicherheit aus beiden Quellen. 

\begin{equation}
    u_{sel} = \frac{1}{n} \sqrt{\frac{\sum_i T_\textit{pixel}^2 - u_\textit{pixel}^2}{n} - \overline T_\textit{sel}^2}
\end{equation}

\begin{equation}
    u_{pixel,cal} = \sqrt{u_{pixel}^2 + u_{sel}^2 + u_\textit{ref}^2}
\end{equation}

\section{Temperaturverlauf}


\end{document}